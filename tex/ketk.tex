\LaTeX{} är inte bara en bra grund att bygga dokument med matematik i. Genom åren har \LaTeX{}:s utomordentliga grund för akademiska texter kommit att inspirera mängder av expansionsbibliotek - vissa är bättre, andra bjuder fortfarande på en hel del buggar - och idag kan man i stort sett hitta stöd för typsättning och layout för vad man än kan tänka sig. Det här avsnittet ämnar inte att behandla detaljer kring några specifika bibliotek. Däremot kommer några få uppslag på vad som finns i avseende, i likhet med kapitel \ref{sec:math}, att belysa möjligheter för att bredda läsarens inspiration.

Om just ditt ämnesområde har ett behov som inte tas upp här eller på annat ställe i detta dokument, tveka inte att söka runt på nätet.

\section{Kemi}
Kemi är ofta en mardröm att typsätta för den som inte är bekant med \LaTeX{}. Det finns två stora bibliotek för att underlätta det hela: \verb?ochem?\footnote{\url{http://www.2k-software.de/ingo/ochem.html}} och \verb?chemfig?. Det här dokumentets sammanställare kan dock inte svara för ochems kapacitet. Chemfig använder \TikZ{} för att producera grafiska representationer på ett lättscriptat vis. För exempel och handledning hänvisas till bibliotekets dokumentation\footnote{\url{http://mirror.ctan.org/macros/latex/contrib/chemfig/chemfig_doc_en.pdf}}.

\section{Elektronik och digitalteknik}
Det finns gott om god extern programvara för att skapa exempelvis kopplingsscheman och för mer komplexa projekt kan dessa oftast vara att föredra i kombination med import av exporterade bilder till \LaTeX{}-dokumentet. För mindre projekt kan dock exempelvis \verb?circuitikz?\footnote{http://http://texdoc.net/texmf-dist/doc/latex/circuitikz/circuitikzmanual.pdf} vara lämpligt. Som namnet antyder bygger det på \TikZ{} och erbjuder ett stort utbud på symboler.

\section{Presentationer och examinationer}
Vant dig vid att fixa allt mellan himmel och jord i \LaTeX{} och vill skapa även dina overhead-slides på WYSIWYM-manér? Ta en titt på \verb?beamer?\footnote{\url{http://www.ctan.org/tex-archive/macros/latex/contrib/beamer/}}\footnote{\url{https://tug.org/TUGboat/tb26-1/mertz.pdf}} och lär dig hur du äntligen kan göra dig kvitt Powerpoint utan att behöva ge upp handout-sammanställningar eller förlita dig på statiska dokument.

Med dokumentklassen \verb?exam? får man tidseffektivt inte bara ett enhetligt utseende som gör eleverna bekanta med en typsättning många av dem kommer att stöta på om och om igen under sina tentamenstillfällen på högskolenivå, man slipper även hålla ordning på två dokument för varje examination (dels examinationen, dels svarsmallar/lösningsexempel). Möjlighet att poängsätta och numrera varje uppgift ingår givetvis, likaså att sömlöst infoga allt annat \LaTeX{} bjuder på.
