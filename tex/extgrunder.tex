\section{Extra grunder}
Innan vi går vidare bör några grundläggande saker tas upp som underlättar markant. Tidigare nämndes bland annat att vissa tecken är reserverade och kräver lite extra för att användas. Det gäller även tecken såsom å,ä, och ö. Det finns många olika sätt att komma förbi det problemet, och jag tänker nämna ett.

I vår preamble kan vi använda oss utav \texttt{\textbackslash usepackage\lbrack utf8\rbrack\{inputenc\}} för att tillåta alla utf-8-kodade tecken som inte är reserverade av \LaTeX. Det gör att vi kan skriva exempelvis ``åäöµ'' i stället för att skriva \texttt{``\tb{}r\{a\}\tb{}``\{a\}\tb{}``\{o\}\$\tb{}mu\$''}.
Det förutsätter dock att våra .tex-filer är teckenkodade för just UTF-8. För Linux- och OSX-användare är det standard, men för Windowsanvändare krävs det att texteditorn i fråga sparar filerna i just UTF-8. Ska filerna endast skrivas under Windows kan ``utf8'' i kommandot ovan bytas mot ``cp1252'', men det rekommenderas generellt sett inte då UTF-8 på många sätt är överlägset och standard i alla moderna sammanhang utanför windowsexklusiva miljöer.

Dessutom vill jag emfasera vikten av att strukturera .tex-filerna ordentligt. Utnyttja kommentarsmöjligheterna. Det är ett utmärkt sätt att skriva ``kom-ihåg-lappar'' till sig själv eller sina medförfattare. Ytterligare en nivå av struktur kan fås genom att använda kommandot \texttt{\tb{}input\{filnamn.tex\}} vilket inkluderar en TeX-fil som om den skrevs direkt i den fil där \tb{}input skrevs. Det innebär att dessa filer inte ska ha någon preamble eller document-environment vilket gör att ``headerfilen'' med preamble, innehållsförteckning etc. kan hållas ren och ge en översiktlig bild över dokumentet i sin helhet medan underfilerna ges en mer lättläst struktur. Dessutom ges då även möjligheten för flera författare att effektivt kunna jobba med olika delar av dokumentet parallellt.

Dessutom bör även återigen emfaseras på filosofin bakom \LaTeX. Att det alls används är just för att det ska underlätta för författaren, inte för att krångla till det. Om det är något som verkar onödigt krånglig även efter att man tagit sig förbi den första tröskeln beror det sannolikt på att man antingen sitter med ett nytt makrobibliotek eller på att man försöker krångla till det mer än nödvändigt. Ställ dig först frågan om det faktiskt är värt mödan att försöka lösa problemet eller om det finns ett närliggande sätt att lösa det på som fungerar fullgott i sammanhanget. Om det är värt mödan har någon säkerligen redan utvecklat en lösning för det, annars är risken att du krånglar till det mer än nödvändigt överhängande ...

Till sist bör även emfaseras på möjligheten att skapa makron, även om det inte är något som kommer att tas upp i någon större utsträckning här. Om någon behöver göras mer än två gånger är det värt att göra ett makro för det. 
